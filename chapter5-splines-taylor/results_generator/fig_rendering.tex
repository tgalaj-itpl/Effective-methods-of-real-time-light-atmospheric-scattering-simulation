\def \my_img_width {0.16} % 1.98/12 = 0.165
\newcommand{\modelrow}[2]{
\rotatebox{90}{\footnotesize #2 /
\input{../output/figures/sza_#2.txt}\unskip\degree} &
\includegraphics[width=\my_img_width \mytablewidth]{#1_#2_nishita93} &
\includegraphics[width=\my_img_width \mytablewidth]{#1_#2_nishita96} &
\includegraphics[width=\my_img_width \mytablewidth]{#1_#2_preetham} &
\includegraphics[width=\my_img_width \mytablewidth]{#1_#2_oneal} &
\includegraphics[width=\my_img_width \mytablewidth]{#1_#2_haber} &
\includegraphics[width=\my_img_width \mytablewidth]{#1_#2_bruneton} &
\includegraphics[width=\my_img_width \mytablewidth]{#1_#2_elek} &
\includegraphics[width=\my_img_width \mytablewidth]{#1_#2_hosek} &
\includegraphics[width=\my_img_width \mytablewidth]{#1_#2_trapezoidal} &
\includegraphics[width=\my_img_width \mytablewidth]{#1_#2_spline} &
\includegraphics[width=\my_img_width \mytablewidth]{#1_#2_taylor} &
\includegraphics[width=\my_img_width \mytablewidth]{#1_#2_libradtran} &
\ifthenelse{\equal{#2}{06h00} \OR \equal{#2}{07h00} \OR \equal{#2}{08h00}}
{\raisebox{0.091\mytablewidth}{\makebox[\my_img_width \mytablewidth]{\footnotesize n/a}}}
{\includegraphics[width=\my_img_width \mytablewidth]{#1_#2_measurements}}}

\newcommand{\modeltable}[3]{
\setlength{\tabcolsep}{0cm}
\renewcommand\arraystretch{0}
\LTcapwidth=\textwidth
\begin{longtable}{p{0.04\mytablewidth}*{14}{p{\my_img_width \mytablewidth}}}
\caption*{#2}\\
&
\footnotesize\bfseries Nishita93 &
\footnotesize\bfseries Nishita96 &
\footnotesize\bfseries Preetham &
\footnotesize\bfseries O'Neal &
\footnotesize\bfseries Haber &
\footnotesize\bfseries Bruneton &
\footnotesize\bfseries Elek &
\footnotesize\bfseries Hosek &
\footnotesize\bfseries Trapezoidal &
\footnotesize\bfseries\leavevmode\color{violet} Spline &
\footnotesize\bfseries\leavevmode\color{orange} Taylor &
\footnotesize\bfseries\leavevmode\color{blue} libRadtran &
\footnotesize\bfseries\leavevmode\color{red} Measurements\vspace{1mm}\\
\modelrow{#1}{06h00}\\
\supplemental{\modelrow{#1}{07h00}\\}
\supplemental{\modelrow{#1}{08h00}\\}
\supplemental{\modelrow{#1}{09h30}\\}
\supplemental{\modelrow{#1}{09h45}\\}
\supplemental{\modelrow{#1}{10h00}\\}
\modelrow{#1}{10h15}\\
\supplemental{\modelrow{#1}{10h30}\\}
\supplemental{\modelrow{#1}{10h45}\\}
\supplemental{\modelrow{#1}{11h00}\\}
\modelrow{#1}{11h15}\\
\supplemental{\modelrow{#1}{11h30}\\}
\supplemental{\modelrow{#1}{11h45}\\}
\supplemental{\modelrow{#1}{12h00}\\}
\supplemental{\modelrow{#1}{12h15}\\}
\supplemental{\modelrow{#1}{12h30}\\}
\supplemental{\modelrow{#1}{12h45}\\}
\supplemental{\modelrow{#1}{13h00}\\}
\modelrow{#1}{13h15}\\
\supplemental{\modelrow{#1}{13h30}\\}
\multicolumn{15}{c}{#3}
\end{longtable}
}

%\myfigure{\modeltable{image_full_spectral}}
\modeltable{image_full_spectral}{%
{\bf Rendering}. Fisheye skydome rendering of the spectral radiance
obtained with each model, convolved with the CIE color matching functions,
converted from XYZ to linear sRGB, and tone mapped with $1-e^{-kL}$, for several
time of day / sun zenith angle values (the red cross indicates the sun
direction). The measurements are interpolated using bicubic spherical
interpolation before rendering. Compare with Fig. 13
in\cite{Kider2014}.\label{fig:rendering}}{\ignorespaces}

