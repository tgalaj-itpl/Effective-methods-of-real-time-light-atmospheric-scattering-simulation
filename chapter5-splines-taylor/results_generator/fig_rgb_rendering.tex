\newcommand{\rgbrenderingcell}[6]{
\begin{overpic}[width=0.198\columnwidth]{image_#3_#2_#1}
\ifthenelse{\equal{#1}{measurements}}{
  %
}{
  \put(5,5){\footnotesize\color{white}%
    \input{program/output/figures/error_#4_#2_#1.txt}}
}
\ifthenelse{\equal{#5}{true}}{
  \put(82,85){\footnotesize\color{white}#6}
}{
  %
}
\end{overpic}}

\newcommand{\rgbrenderingrow}[4]{
\rgbrenderingcell{nishita93}{#1}{#2}{#3}{#4}{\hspace{1.4mm}3} &
\rgbrenderingcell{nishita96}{#1}{#2}{#3}{#4}{\hspace{1.4mm}3} &
\rgbrenderingcell{preetham}{#1}{#2}{#3}{#4}{\hspace{1.4mm}3} &
\rgbrenderingcell{oneal}{#1}{#2}{#3}{#4}{\hspace{1.4mm}3} &
\rgbrenderingcell{haber}{#1}{#2}{#3}{#4}{\hspace{1.4mm}8} &
\rgbrenderingcell{bruneton}{#1}{#2}{#3}{#4}{\hspace{1.4mm}3} &
\rgbrenderingcell{elek}{#1}{#2}{#3}{#4}{15} &
\rgbrenderingcell{hosek}{#1}{#2}{#3}{#4}{11} &
\rgbrenderingcell{libradtran}{#1}{#2}{#3}{#4}{40} &
\rgbrenderingcell{measurements}{#1}{#2}{#3}{#4}{40}}

\newcommand{\rgbdiffcell}[2]{
\begin{overpic}[width=0.198\columnwidth]{image_approx_diff_#2_#1}
\put(5,5){\footnotesize\color{white}%
  \input{program/output/figures/image_approx_psnr_#2_#1.txt}dB}
\end{overpic}}

\newcommand{\rgbdiffrow}[1]{
\rgbdiffcell{nishita93}{#1} & 
\rgbdiffcell{nishita96}{#1} &
\rgbdiffcell{preetham}{#1} &
\rgbdiffcell{oneal}{#1} &
\rgbdiffcell{haber}{#1} &
\rgbdiffcell{bruneton}{#1} &
\rgbdiffcell{elek}{#1} &
\rgbdiffcell{hosek}{#1} &
\rgbdiffcell{libradtran}{#1} &
\rgbdiffcell{measurements}{#1}}

\begin{figure*}
\begin{center}
\begin{mytabular}{p{0.04\columnwidth}*{10}{p{0.198\columnwidth}}}
&
\footnotesize\bfseries Nishita93 &
\footnotesize\bfseries Nishita96 &
\footnotesize\bfseries Preetham &
\footnotesize\bfseries O'Neal &
\footnotesize\bfseries Haber &
\footnotesize\bfseries Bruneton &
\footnotesize\bfseries Elek &
\footnotesize\bfseries Hosek &
\footnotesize\bfseries\leavevmode\color{blue} libRadtran &
\footnotesize\bfseries\leavevmode\color{red} Measurements\vspace{1mm}\\
\rotatebox{90}{\footnotesize original} & 
\rgbrenderingrow{10h15}{original}{original_rgb}{true}\\
\rotatebox{90}{\footnotesize \cref{eq:spectraltorgb}} & 
\rgbrenderingrow{10h15}{full_spectral}{rgb}{false}\\
\rotatebox{90}{\footnotesize \cref{eq:rgb}} & 
\rgbrenderingrow{10h15}{approx_spectral}{approximate_rgb}{false}\\
\rotatebox{90}{\footnotesize $10|\mathrm{(1)}-\mathrm{(2)}|$} &
\rgbdiffrow{10h15}
\end{mytabular}
\end{center}
\caption{{\bf Spectrum sampling}. Rendering of the skydome by using the number
of wavelengths $n_\lambda$ proposed by the authors of each model (1st row,
$n_\lambda$ in the top right), by using $n_\lambda=40$ and
\cref{eq:spectraltorgb} (2nd row), and by using $n_\lambda=3$ and \cref{eq:rgb}
(3rd row). The bottom left number in each cell is the RMSE, times 1000, compared
to the top right cell (summed over the 81 measurement samples, on the RGB colors
tone-mapped to the $[0,1]$ interval). The 4th row is the difference between the
2nd and 3rd ones, times 10, with the corresponding peak signal to noise ratio
(PSNR).}\label{fig:rgb_rendering}
\end{figure*}
